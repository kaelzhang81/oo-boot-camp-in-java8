\section{OO Boot Camp Explained}
\label{sec:oo-camp}

\begin{frame}
  \begin{center}
    \Huge{\textcolor{red}{Boot Camp Explained}}
  \end{center}
\end{frame}

\subsection{迭代1}

\begin{frame}{需求1}
  \begin{block}{规则:长度单位的比较运算}
    \begin{enumerate}
      \item 3 Mile == 3 Mile
      \item 3 Mile != 2 Mile      
    \end{enumerate}
  \end{block}
\end{frame}

\subsection{迭代2}

\begin{frame}{需求2}
  \begin{block}{规则:支持Mile与Yard之间的混合比较}
    \begin{enumerate}
      \item 1 Mile == 1760 Yard
      \item 3 Yard == 3 Yard
      \item 1 Mile != 1761 Yard
      \item 3 Yard != 4 Yard
    \end{enumerate}
  \end{block}
\end{frame}

\subsection{迭代3}

\begin{frame}{需求3}
  \begin{block}{规则:支持Mile, Yard, Feet, Inch的混合比较}
    \begin{enumerate}
      \item 1 Yard == 3 Feet
      \item 1 Feet == 12 Inch
    \end{enumerate}
  \end{block}
\end{frame}

\subsection{迭代4}

\begin{frame}{需求4}
  \begin{block}{规则:支持四则混合运算}
    \begin{enumerate}
      \item 13 Inch + 11 Inch == 2 Feet
      \item 2 Yard - 3 Feet == 1 Yard
    \end{enumerate}
  \end{block}
\end{frame}

\subsection{迭代5}

\begin{frame}{需求5}
  \begin{block}{规则: 增加容量单位体系}
    \begin{enumerate}
      \item 1 TBSP == 3 TSP
      \item 1 OZ == 2 TBSP
    \end{enumerate}
  \end{block}

  \begin{block}{约束}
    \begin{enumerate}
      \item 可以对比任意两个容量的相等性
      \item 只允许用户使用现有的三个容量单位来表示容量
      \item 两个容量可以混合四则运算
    \end{enumerate}
  \end{block}
\end{frame}

\subsection{迭代6}

\begin{frame}{需求6}
  \begin{block}{输出规则1}
    \begin{enumerate}
      \item 以Inch为单位输出任何Length对象
      \item 数量和单位之间以一个空格分隔
    \end{enumerate}
  \end{block}

  \begin{block}{例子}
    \begin{enumerate}
      \item 2 FEET => 24 INCH
      \item 2 YARD => 72 INCH
    \end{enumerate}
  \end{block}
\end{frame}

\subsection{迭代7}

\begin{frame}{需求7}
  \begin{block}{输出规则2}
    \begin{enumerate}
      \item 如果一个对象的“基准单位数量”在一个更大的单位上的倍数非0,则显示此对象在此单位上的倍数,以及此单位的名字;
      \item 如果一个对象的“基准单位数量”在一个更大的单位上的倍数为0,则无须显示此对象在此单位上的倍数,以及此单位的名字;
      \item 如果余数在一个较小的单位上倍数非0,则显示此对象在此单位上的倍数, 以及此单位的名字;
      \item 如果余数在一个较小的单位上倍数为0,则无须显示此对象在此单位上的倍数,以及此单位的名字;
      \item 如果存在多个“数量+单位”组合,则按照单位大小,从左向右排列;
      \item 数量和单位之间,由一个空格分开;“数量+单位”之间由一个空格分开。
    \end{enumerate}
  \end{block}
\end{frame}

\begin{frame}{需求7}
  \begin{block}{例子}
    \begin{enumerate}
      \item 14 INCH => 1 FEET 2 INCH
      \item 24 INCH => 2 FEET
      \item 39 INCH => 1 YARD 3 INCH
      \item 1762 YARD => 1 MILE 2 INCH
    \end{enumerate}
  \end{block}
\end{frame}

