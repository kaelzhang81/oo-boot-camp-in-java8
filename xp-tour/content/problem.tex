\section{提出问题}
\label{sec:problem}

\begin{frame}
  \begin{center}
    \Huge{\textcolor{red}{提出问题}}
  \end{center}
\end{frame}

\subsection{提出问题}

\begin{frame}[fragile]{需求}
  \begin{block}{老师说出3个特殊数,例如3, 5, 7,让100个学生依次报数}
  \begin{enumerate}
    \item 如果所报数字是「第一个特殊数(3)」的倍数时说Fizz;如果所报数字是「第二个特殊数(5)」的倍数时说Buzz;如果所报数字是「第三个特殊数(7)」的倍数时说Whizz;
    \item 如果所报数字同时是「两个特殊数」的倍数,也要特殊处理。例如,如果是「第一个(3)」和「第二个(5)」特殊数的倍数,那么也不能说该数字,而是要说FizzBuzz。以此类推,如果同时是三个特殊数的倍数,那么要说FizzBuzzWhizz;
    \item 如果所报数字包含了「第1个(3)」特殊数时,忽略规则1和2,直接说Fizz。例如,要报13的同学应该说Fizz;要报35,它既包含3,同时也是5和7的倍数,要说Fizz,而不能说BuzzWhizz;
    \item 否则,直接说出要报的数字。
  \end{enumerate}
  \end{block}  
\end{frame}

\begin{frame}[fragile]{形式化}
  \begin{c++}
r1:
- times(3) -> Fizz
- times(5) -> Buzz
- times(7) -> Whizz
r2:
- times(3) && times(5) && times(7) -> FizzBuzzWhizz
- times(3) && times(5) -> FizzBuzz
- times(3) && times(7) -> FizzWhizz
- times(5) && times(7) -> BuzzWhizz
r3:
- contains(3) -> Fizz
- the priority of contains(3) is highest
rd:
- num -> "num"
  \end{c++}
\end{frame}
